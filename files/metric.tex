\section{Basic Notions in Metric Geometry}

We start from intrinsic metric spaces.
The authors choose the equivalent definition of existence of $\delta$-midpoint, however, we use the more basic definition.

\begin{nota}
    Let $X$ be a metric space, the distance between $x,y\in X$ will be denoted by $|xy|$.
\end{nota}

\begin{defn}
    Let $X$ be a metric space, $\gamma:[a,b]\to X$ be a curve.
    We define the \emph{length} of $\gamma$ to be
    \[L[\gamma]:=\sup\left\{\sum_{i=1}^n|x_{i-1}x_i|\colon a=x_0<x_1<\cdots<x_{n_1}<x_n=b\right\}.\]
    If $L[\gamma]<+\infty$, we call $\gamma$ is \emph{rectifiable}.
    Denote the set of all the (isomorphism classes under linear reparametrization of) rectifiable curves between $x,y$ by $\Omega_{xy}$.
\end{defn}

\begin{defn}
    Let $X$ be a metric space.
    If for any $x,y\in X$, we have
    \[|xy|=\inf_{\gamma\in\Omega_{xy}}L[\gamma],\]
    then we call $X$ an \emph{intrinsic metric space}.
\end{defn}

\begin{prop}\label{prop:delta midpoint}
    Let $X$ be a complete metric space, then $X$ is intrinsic if and only if for any $x,y\in X$ and $\delta>0$, there exists $z\in X$, such that
    \[|xz|<\frac{1}{2}|xy|+\delta,\ |yz|<\frac{1}{2}|xy|+\delta.\]
\end{prop}
\begin{proof}
    The only if part is relatively easy, we only show if part.
    Fix $\delta>0$.
    For $x,y$, we define a curve $\gamma:[0,1]\to X$ with $L[\gamma]<|xy|+\delta$.
    By assumption, we can choose $\gamma(1/2)$ with
    \[\left|x\gamma\left(\frac{1}{2}\right)\right|<\frac{1}{2}|xy|+\frac{1}{2^2}\delta,\ \left|y\gamma\left(\frac{1}{2}\right)\right|<\frac{1}{2}|xy|+\frac{1}{2^2}\delta.\]
    Then we choose
    \begin{gather*}
        \left|x\gamma\left(\frac{1}{4}\right)\right|<\frac{1}{2}\left|x\gamma\left(\frac{1}{2}\right)\right|+\frac{1}{4^2}\delta,\ \left|\gamma\left(\frac{1}{4}\right)\gamma\left(\frac{1}{2}\right)\right|<\frac{1}{2}\left|x\gamma\left(\frac{1}{2}\right)\right|+\frac{1}{4^2}\delta\\
        \left|\gamma\left(\frac{3}{4}\right)\gamma\left(\frac{1}{2}\right)\right|<\frac{1}{2}\left|\gamma\left(\frac{1}{2}\right)y\right|+\frac{1}{4^2}\delta,\ \left|\gamma\left(\frac{3}{4}\right)y\right|<\frac{1}{2}\left|x\gamma\left(\frac{1}{2}\right)\right|+\frac{1}{4^2}\delta.
    \end{gather*}
    Define inductively to all diadic numbers, add $\gamma(0)=x,\gamma(1)=y$.
    Notice that $\gamma$ is Lipschitz, hence $\gamma$ can be defined on whole $[0,1]$.
    Since
    \begin{align*}
        \sum_{i=0}^{2^n-1}\left|\gamma\left(\frac{i}{2^n}\right)\gamma\left(\frac{i+1}{2^n}\right)\right|&<\sum_{i=0}^{2^{n-1}-1}\left(\left|\gamma\left(\frac{i}{2^{n-1}}\right)\gamma\left(\frac{i+1}{2^{n-1}}\right)\right|+\frac{1}{2^{2n-1}}\delta\right)\\
        &<\cdots\\
        &<|xy|+\delta,
    \end{align*}
    passing to limit, we obtain $L[\gamma]<|xy|+\delta$.
\end{proof}

\begin{defn}
    A \emph{geodesic} is a curve whose length is equal to the distance between its ends.
    An intrinsic metric space is called \emph{geodesic} if any two points can be joined with a geodesic.
\end{defn}

\begin{nota}
    We use $[xy]$ to denote a geodesic between $x$ and $y$.
    There may be several geodesics between $x$ and $y$, and if we use this notation, we mean we specified a particular geodesic.
\end{nota}

\begin{prop}\label{prop:midpoint}
    Let $X$ be a complete metric space, then $X$ is geodesic if and only if for any $x,y\in X$, there exists $z\in X$, such that
    \[|xz|=|yz|=\frac{1}{2}|xy|.\]
\end{prop}
\begin{proof}
    We also only show if part, and this can be achieved by taking $\delta=0$ in the proof of Proposition~\ref{prop:delta midpoint}.
\end{proof}

We now discuss the Hopf--Rinow theorem.
The proof is taken from~\cite[2.15]{alexanderAlexandrovGeometry2024}.

\begin{defn}
    A metric space $X$ is called \emph{proper} if any closed bounded subset of $X$ is compact.
\end{defn}

\begin{lem}
    Proper intrinsic metric spaces are geodesic.
\end{lem}
\begin{proof}
    Let $X$ be a proper intrinsic metric space.
    Consider $1/n$-midpoints $z_n$ of two points $x,y\in X$.
    Since they are contained in the ball $B(x,|xy|)$, they are contained in $\overline{B(x,|xy|)}$, which is bounded and closed, hence compact.
    Therefore $\{z_n\}$ contains a convergent subsequence, which converge to the midpoint of $x$ and $y$, that is, the midpoint of $x$ and $y$ exists.
    By Proposition~\ref{prop:midpoint}, $X$ is geodesic.
\end{proof}

\begin{thm}[Hopf--Rinow]
    Locally compact complete intrinsic metric spaces are proper.
\end{thm}
\begin{proof}
    Let $X$ be a locally compact complete intrinsic metric space.
    For $x\in X$, define $\rho(x)$ to be the supremum of all $R>0$ such that $\overline{B(x,R)}$ is compact.
    Since $X$ is locally compact, $\rho(x)>0$ for any $x\in X$.
    It's sufficient to show $\rho(x)=+\infty$ for some (and therefore any) $x$.
    Suppose not, i.e., $\rho(x)<+\infty$.

    First, notice that $B=\overline{B(x,\rho(x))}$ is compact.
    Indeed, for any $\varepsilon>0$, $\overline{B(x,\rho(x)-\varepsilon)}$ is compact, and since $X$ is intrinsic, it is an $\varepsilon$-net of $B$, hence $B$ is compact.

    Second, we claim that $|\rho(x)-\rho(y)|\leq|xy|$, and in particular, $\rho$ is continuous.
    If this does not hold, we have $\rho(x)+|xy|<\rho(y)$.
    Then $\overline{B(x,\rho(x)+\varepsilon)}$ is a closed subset of $\overline{B(y,\rho(y))}$ for sufficiently small $\varepsilon>0$, hence $\overline{B(x,\rho(x)+\varepsilon)}$ is compact.
    This contradicts to the definition of $\rho$.

    Now let $\varepsilon=\min_{y\in B}\rho(y)$, since $B$ is compact, the minimum can be reached, and $\varepsilon>0$.
    Choose a finite $\varepsilon/10$-net $\{a_1,\cdots,a_n\}$ in $B$, set
    \[W=\bigcup_{i=1}^n\overline{B(a_i,\varepsilon)}.\]
    Then $W$ is compact.
    However, $\overline{B(x,\rho(x)+\varepsilon/10)}\subset W$, this means $\overline{B(x,\rho(x)+\varepsilon/10)}$ is compact.
    This contradicts to the definition of $\rho$.
    Hence we must have $\rho(x)=+\infty$.
\end{proof}

\begin{cor}
    Locally compact complete intrinsic metric spaces are geodesic.
\end{cor}

\begin{rem}
    After stating Hopf--Rinow theorem in 2.1, \cite{buragoADAlexandrovSpaces1992}~claims that the limit of geodesics is still a geodesic.
    This is not correct.
    Think about the surface of a solid cylinder, consider the geodesics connecting antipodal points converging to the upper face.
    They converge to a semicircle, but it is not a geodesic---now the shortest path is a line segement on the upper face.
\end{rem}

\begin{defn}
    Let $X$ be an intrinsic metric space.
    A subset $S\subset X$ is called \emph{convex} if any two points $p,q\in S$ can be joined with a geodesic, and the geodesic $[pq]\subset S$.
\end{defn}

\begin{rem}
    The definition of convexity in ambiguous in~\cite{buragoADAlexandrovSpaces1992}, so we adopt the usual definition.
\end{rem}

\begin{defn}
    A \emph{triangle} on an intrinsic metric space $X$ is the collection of three points $p,q,r\in X$ and three geodesics $[qr],[pr],[pq]$, denoted by $\triangle pqr$.
\end{defn}

\begin{nota}
    Denote $S^2_k$ the $2$-dimensional complete simply-connected Riemannian manifold of curvature $k$.
\end{nota}

\begin{defn}
    Let $p,q,r$ be a triple of points in an intrinsic metric space $X$.
    Define their \emph{comparison triangle on $S^2_k$}, denoted by $\tilde{\triangle}_kpqr$, is the triangle with vertices $\tilde{p},\tilde{q},\tilde{r}$ on $S^2_k$ with $|pq|=|\tilde{p}\tilde{q}|$, $|pr|=|\tilde{p}\tilde{r}|$, $|qr|=|\tilde{q}\tilde{r}$.
    Comparison triangle always exists up to a rigid motion when $k\leq 0$, and when $k>0$ we ask the perimeter of $\triangle{pqr}$ is less than $2\pi/\sqrt{k}$.
    We denote $\tilde{\measuredangle}_kpqr$ by the angle at vertex $\tilde{q}$ in triangle $\tilde{\triangle}_kpqr$.
\end{defn}