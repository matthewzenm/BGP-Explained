\section{Basic Concepts}
\begin{defn}
    A locally complete space $X$ with intrinsic metric is called \emph{space with curvature $\geq k$} if in some neighborhood $U_x$ of each point $x\in X$ the following condition is satisfied:
    \begin{itemize}
        \item[(D)] for any four distinct points $(a;b,c,d)$ in $U_x$ we have the inequality
        \[\tilde{\measuredangle}^kbac+\tilde{\measuredangle}^kbad+\tilde{\measuredangle}^kcad\leq 2\pi.\]
    \end{itemize}
    If $X$ is a $1$-dimensional manifold and $k>0$, we require in addition that $\diam{X}\leq\pi/\sqrt{k}$.
\end{defn}

For traditional Toponogov's comparison theorem, we have the following condition.
\begin{thm}\label{thm:CBB A condition}
    A geodesic space $X$ is a space with curvature $\geq k$ if and only if the following condition is satisfied:
    \begin{itemize}
        \item[\rm (A)] for any triangle $\triangle{pqr}$ with vertices in $U_x$ and any point $s\in[qr]$, the inequality $|ps|\geq|\tilde{p}\tilde{s}|$ is satisfied, where $\tilde{s}$ is the point on the side $[\tilde{q}\tilde{r}]$ of the comparison triangle $\tilde{\triangle}^kpqr$ with $|qs|=|\tilde{q}\tilde{s}|$.
    \end{itemize}
\end{thm}

To prove the only if part, we need the following lemma.

\begin{lem}[Alexandrov]\label{lem:Alexandrov}
    Let $\tilde{\triangle}^kpqs,\tilde{\triangle}^kprs$ be given on $S^2_k$, which are joined to each other in an exterior way along the side $[ps]$.
    Let there also be given $\tilde{\triangle}bcd$, where $|bc|=|pq|$, $|bd|=|pr|$, $|cd|=|qs|+|rs|$, and $|bc|+|bd|+|cd|<2\pi/\sqrt{k}$ in the case $k>0$.
    Then $\tilde{\measuredangle}^kpsq+\tilde{\measuredangle}^kpsr\leq\pi$ ($\geq\pi$) if and only if $\tilde{\measuredangle}^kpqs\geq\tilde{\measuredangle}^kbcd$ and $\tilde{\measuredangle}^kprs\geq\tilde{\measuredangle}^kbdc$ (respectively $\tilde{\measuredangle}^kpqs\leq\tilde{\measuredangle}^kbcd$ and $\tilde{\measuredangle}^kprs\leq\tilde{\measuredangle}^kbdc$).
\end{lem}
\begin{proof}
    Observe that in absolute geometry (Euclidean, spherical or non-Euclidean), when two sides are fixed, bigger angle is opposite to bigger side.
    Using a rigid motion, we can assume $q=c$ and $s\in[cd]$.
    Then $\tilde{\measuredangle}^kpsq+\tilde{\measuredangle}^kpsr\leq\pi$ is equivalent to $\tilde{\measuredangle}^krsd\leq\tilde{\measuredangle}^kpsd$.
    Using observation, this is equivalent to $|pd|\geq|pr|=|bd|$.
    Using observation again, this is equivalent to $\tilde{\measuredangle}^kpqs\geq\tilde{\measuredangle}^kbcd$.
    The other inequality holds similarly.
\end{proof}

\begin{proof}[Proof of $\mathrm{(D)}\implies\mathrm{(A)}$]
    Apply Alexandrov lemma (Lemma~\ref{lem:Alexandrov}) to comparison triangle $\tilde{\triangle}^kpqs$ and $\tilde{\triangle}^kpsr$.
    By (D), we have $\tilde{\measuredangle}^kpsq+\tilde{\measuredangle}^kpsr\leq\pi$ (since $\tilde{\measuredangle}^kqsr=\pi$), hence $\tilde{\measuredangle}^kpqs\geq\tilde{\measuredangle}^kpqr$, that is, $|ps|\geq|\tilde{p}\tilde{s}|$.
\end{proof}

The proof of converse needs more work.
We need the notion of angle between two geodesics with the same origin.

\begin{nota}
    Let $\gamma,\sigma$ be geodesics with origin $p$, $q,r$ on $\gamma,\sigma$ respectively and $x=|pq|,y=|pr|$.
    Denote $\omega_k(x,y)=\tilde{\measuredangle}^kqpr$.
\end{nota}

\begin{defn}\label{defn:angle}
    Let $\gamma,\sigma$ be geodesics with origin $p$.
    If the limit
    \[\lim_{x,y\to 0}\omega_k(x,y)\]
    exists, the limit is called the \emph{angle} between $\gamma$ and $\sigma$.
\end{defn}

We need to show angle between two geodesics does not depend on the choice of $k$.

\begin{lem}
    Let $p,q,r$ be three points in a metric space, then $k\mapsto\tilde{\measuredangle}^kqpr$ is increasing.
\end{lem}
\begin{proof}
    This follows from the triangle version of the Toponogov's comparison theorem (cf.~\cite[Theorem 12.2.2]{petersenRiemannianGeometry2016}).
    Since angles on Riemannian manifolds are defined by inner product, this proof does not rely on Definition~\ref{defn:angle}.
\end{proof}

\begin{lem}
    Let $p,q,r\in X$, with $|pq|+|qr|+|rp|<2\pi/\sqrt{k}$ when $k>0$.
    Let $\ell=\max\{|pq|,|pr|\}$, then we have
    \[A[\tilde{\triangle}pqr]\leq\pi(\ell^2+o(\ell^2)).\]
\end{lem}
\begin{proof}
    We adopt geodesic polar coordinate.
    Since geodesic balls are convex, we have
    \begin{align*}
        A[\tilde{\triangle}pqr]&\leq\int_{0}^{\ell}\int_{0}^{\theta}\sn^k(r)\ ds\ dr\\
        &=\theta\md^k(\ell)\\
        &\leq\pi\md^k(\ell)\\
        &=\pi(\ell^2+o(\ell^2)).
    \end{align*}
    Here $\sn^k{}$ and $\md^k{}$ are solutions to initial value problem
    \[\sn^k(t)+k(\sn^k)''(t)=0,\ \sn^k(0)=0,\ (\sn^k)'(0)=1,\]
    and
    \[\md^k(t)+k(\md^k)''(t)=1,\ \md^k(0)=0,\ (\md^k)'(0)=0.\qedhere\]
\end{proof}

\begin{lem}
    Let $p,q,r\in X$, with $|pq|+|qr|+|rp|<2\pi/\sqrt{k}$ when $k>0$.
    Let $\ell=\max\{|pq|,|pr|\}$, $k<K$, then we have
    \begin{equation}
        0<\tilde{\measuredangle}^Kqpr-\tilde{\measuredangle}^kqpr\leq\pi(|K|+|k|)(\ell^2+o(\ell^2)).\label{eq:lem angle}
    \end{equation}
\end{lem}
\begin{proof}
    By previous lemmas and Gauss--Bonnet formula, we have
    \begin{align*}
        0&<\tilde{\measuredangle}^Kqpr-\tilde{\measuredangle}^kqpr\\
        &<\tilde{\measuredangle}^Kqpr+\tilde{\measuredangle}^Kpqr+\tilde{\measuredangle}^Kprq-\tilde{\measuredangle}^kqpr-\tilde{\measuredangle}^kpqr-\tilde{\measuredangle}^kprq\\
        &=KA[\tilde{\triangle}^Kpqr]-kA[\tilde{\triangle}^kpqr]\\
        &\leq(|K|+|k|)(\ell^2+o(\ell^2)).\qedhere
    \end{align*}
\end{proof}

Let $\ell\to 0$ in inequality~\eqref{eq:lem angle}, we have thus shown the angle between geodesics is well-defined.

Let us back to our discussion on spaces with curvature bounded below.
(A) has the following corollary.
\begin{prop}\label{prop:CBB B condition}
    If (A) is satisfied, then for any geodesics $\gamma,\sigma$ with origin $p$, the function $\omega_k(x,y)$ is non-increasing with respective to each variable $x,y$ when $x,y$ are sufficiently small.
\end{prop}
\begin{proof}
    Let $U_p$ be a neighborhood of $p$ such that (A) holds, $x,x'\in\gamma$, $y\in\sigma$ and $x$ is between $p$ and $x'$.
    We must show $\omega_k(x,y)\geq\omega_k(x',y)$.
    By (A), we have $|yx|\geq|\tilde{y}\tilde{x}|$, then we have
    \[\omega_k(x,y)=\tilde{\measuredangle}^kypx\geq\measuredangle{\tilde{y}\tilde{p}\tilde{x}}=\tilde{\measuredangle}^kypx'=\omega_k(x',y).\]
    Similarly $\omega_k(x,y)$ is non-increasing in $y$.
\end{proof}

By Proposition~\ref{prop:CBB B condition}, the angle between two geodesics with same origin is always defined.

\begin{prop}\label{prop:angle triangle inequality}
    The angles between three geodesics with same origin satisfy triangle inequality.
\end{prop}
\begin{proof}
    A proof for general metric space can be found in~\cite[6.5]{alexanderAlexandrovGeometry2024}.
    But for spaces with curvature $\geq k$, notice that angles in $S^2_k$ satisfy triangle inequality, and then we can take the limit.
\end{proof}

Proposition~\ref{prop:CBB B condition}~also has the following direct corollaries.
\begin{cor}
    If {\rm (A)} is satisfied, then
    \begin{itemize}
        \item[\rm (C)] For any triangle $\triangle{pqr}$ contained in $U_x$, none of its angles is less than the corresponding angle of the comparison triangle $\tilde{\triangle}^kpqr$ in $S^2_k$.
        \item[\rm (C\textsubscript{1})] If $r$ is an interior point of a geodesic $[pq]$, then for any geodesic $[rs]$, we have $\measuredangle{prs}+\measuredangle{qrs}=\pi$. 
    \end{itemize}
\end{cor}

Now we are ready to prove $\mathrm{(A)}\implies\mathrm{(D)}$.
\begin{proof}[Proof of $\mathrm{(A)}\implies\mathrm{(D)}$]
    Let $(p;a,b,c)$ in $U_x$.
    Choose $d\in[pa]$ closed to $p$, then by triangle inequality (Proposition~\ref{prop:angle triangle inequality}) and (C\textsubscript{1}) we have
    \begin{align*}
        \measuredangle{adb}+\measuredangle{bdc}+\measuredangle{cda}&\leq\measuredangle{adb}+\measuredangle{bdp}+\measuredangle{pdc}+\measuredangle{cda}\\
        &=2\pi.
    \end{align*}
    Apply Proposition~\ref{prop:CBB B condition}, we have
    \begin{align*}
        \tilde{\measuredangle}^kadb+\tilde{\measuredangle}^kbdc+\tilde{\measuredangle}^kcda&\leq\measuredangle{adb}+\measuredangle{bdc}+\measuredangle{cda}\\
        &\leq 2\pi.
    \end{align*}
    Since angles vary continuously in $S^2_k$, let $d\to p$, we obtain
    \[\tilde{\measuredangle}^kapb+\tilde{\measuredangle}^kbpc+\tilde{\measuredangle}^kcpa\leq 2\pi.\qedhere\]
\end{proof}